\section{Problem Statement}
\label{sec:problem_statement}

The current cloud computing landscape is dominated by a handful of large providers, including Amazon Web Services (AWS), Google Cloud Platform (GCP), and Microsoft Azure, which together control roughly 70\% of the cloud computing market, as of 2024. While these providers offer a comprehensive range of services spanning basic compute and storage to advanced machine learning and analytics tools, this centralized model presents several significant challenges:

\begin{itemize}
    \item \textbf{Vendor lock-in:} The choice of a cloud provider often becomes a long-term commitment due to:
    \begin{itemize}
        \item Proprietary APIs and services that are not easily portable
        \item Substantial costs and technical challenges associated with migrating data and applications
        \item Pricing models that incentivize increased usage within a single ecosystem
    \end{itemize}

    \item \textbf{Limited control and transparency:} Users face significant constraints in managing their cloud resources:
    \begin{itemize}
        \item Lack of granular control over data storage locations and processing methods
        \item Limited visibility into the underlying infrastructure and its performance characteristics
        \item Difficulty in optimizing applications due to the abstracted nature of cloud services
    \end{itemize}

    \item \textbf{Data privacy and security concerns:} Centralized cloud providers present attractive targets for cyberattacks:
    \begin{itemize}
        \item A single breach can potentially compromise vast amounts of data across multiple clients
        \item Users must trust the provider's security measures without full transparency
        \item Compliance with varying international data protection regulations becomes complex
    \end{itemize}

    \item \textbf{Cost and resource inefficiencies:} The pricing structures of major cloud providers can lead to financial inefficiencies:
    \begin{itemize}
        \item Complex, often opaque pricing models make cost prediction and management challenging
        \item Users frequently over-provision resources to ensure performance, leading to waste
        \item Lack of a true pay-per-use model for many services results in unnecessary expenses
    \end{itemize}
\end{itemize}

These issues have varying impacts across different stakeholders in the cloud computing ecosystem:

\begin{itemize}
    \item \textbf{Individual Developers:} Face barriers to experimentation and innovation due to high costs and complex management of cloud resources. They may struggle to scale their projects efficiently as requirements grow.

    \item \textbf{Startups:} Often find themselves locked into a specific provider early in their development, limiting future flexibility. They may face sudden cost increases as they scale, impacting their financial planning and viability.

    \item \textbf{Enterprises:} Struggle with vendor lock-in, which can impact long-term strategic decisions. They face significant challenges in maintaining regulatory compliance across different regions and in protecting sensitive data.

    \item \textbf{Government and Non-Profit Organizations:} May face difficulties in ensuring data sovereignty and meeting strict security requirements. Budget constraints can limit their ability to leverage advanced cloud services effectively.
\end{itemize}

Addressing these challenges is crucial for fostering a more open, efficient, and innovative cloud computing ecosystem. The need for a solution that provides greater flexibility, transparency, and cost-effectiveness while maintaining robust security and performance is evident across all sectors of the industry.
