\section{Proposed Solution}
\label{sec:proposed_solution}

To address the limitations of the current centralized cloud computing model, we propose a novel decentralized cloud platform that combines the best aspects of traditional cloud services with the advantages of decentralized systems. Our solution is designed to tackle each of the key issues identified in the problem statement while introducing innovative features that enhance flexibility, security, and efficiency.

\subsection{Core Components of the Solution}

\subsubsection{Decentralized Infrastructure}
Our platform leverages a network of independent node providers, each capable of hosting and executing arbitrary workloads. This approach:
\begin{itemize}
    \item Mitigates vendor lock-in by offering a diverse ecosystem of providers
    \item Enhances resilience by distributing resources across nodes of different providers
    \item Improves resource utilization by tapping into underused computational capacity
\end{itemize}

\subsubsection{Reputation-Based Trust System}
\label{subsec:reputation_system}
Unlike typical Web3 solutions that rely on consensus mechanisms, our platform implements a robust reputation system:
\begin{itemize}
    \item Node providers and developers build and maintain long-term reputations
    \item Users can select providers based on their track record of reliability and performance
    \item The system incentivizes honest behavior and high-quality service
\end{itemize}

This approach addresses the trust and security concerns associated with decentralized systems while maintaining the performance advantages of traditional cloud services.

\subsubsection{Efficient Resource Allocation}
Our platform can be used to ensure optimal resource allocation. For instance:
\begin{itemize}
    \item Task requirements can be matched with available node resources in real-time
    \item Dynamic load balancing optimizes resource utilization across the network
    \item Pricing is transparent and competitive, driven by market demand
\end{itemize}

This approach allows users to tackle the cost inefficiencies and opaque pricing models of traditional centralized providers.

\subsubsection{Enhanced Security and Privacy}

The system supports sensitive data processing and storage using Confidential Computing VMs, and also facilitates the rental of GPU nodes for Machine Learning (ML) and Artificial Intelligence (AI) training and inference applications at reasonable market prices, catering to different developer preferences. Some developers may opt for the lowest cost GPUs irrespective of node provider reputation and confidentiality guarantees, while others may prefer high-end GPUs with higher node provider reputation. This flexibility accommodates all developer categories, a feature unique to this platform.

For some use cases, such as storing and rotating tokens, certificates, and other secrets, developers will need nodes with additional security. For these use cases, platform will support Confidential Containers\cite{brasser2022trusted}.

To address data privacy and security concerns, our platform offers:
\begin{itemize}
    \item Support for Confidential Computing VMs, enabling secure processing of sensitive data
    \item End-to-end encryption for data in transit and at rest
    \item Granular control over data location and processing parameters
\end{itemize}

\subsection{Addressing Specific Challenges}

\subsubsection{Overcoming Vendor Lock-in}
Our platform's open architecture and standardized interfaces allow for easy migration between providers, addressing the vendor lock-in issue prevalent in centralized systems.

\subsubsection{Improving Control and Transparency}
Users have unprecedented visibility into the underlying infrastructure and can choose specific nodes or node providers based on their requirements, enhancing control and transparency.

\subsubsection{Enhancing Cost-Effectiveness}
The platform's market-driven pricing model and efficient resource allocation system ensure that users only pay for the resources they actually use, addressing the cost inefficiencies of traditional cloud services.

\subsection{Innovative Features}

\subsubsection{Wide Range of Applications}
Our platform supports diverse applications, from scientific high-performance computing and machine learning to web hosting and data storage, catering to a broad spectrum of user needs.

\subsubsection{DAO Governance}
We implement a Decentralized Autonomous Organization (DAO) model for platform governance, ensuring transparency and community-driven decision-making.

\subsubsection{Blockchain-Based Control Plane}
By running the control plane on a blockchain, we ensure transparent and immutable record-keeping for financial transactions and reputation tracking, while keeping the performance-sensitive data plane on the regular internet for optimal speed.

\subsection{Benefits for Stakeholders}

\subsubsection{For Developers}
\begin{itemize}
    \item Access to a diverse range of computational resources
    \item Flexibility to choose providers based on specific needs
    \item Potential for significant cost savings
\end{itemize}

\subsubsection{For Node Providers}
\begin{itemize}
    \item Opportunity to monetize unused computational resources
    \item Access to a global market of users
    \item Incentives for maintaining high-quality service through the reputation system
\end{itemize}

\begin{figure*}[h]
    \centering
    \includegraphics[width=\textwidth]{figures/proposed_architecture.png}
    \caption{Proposed platform architecture.}
    \label{fig:proposed-architecture}
\end{figure*}

Figure~\ref{fig:proposed-architecture} illustrates the architecture of our proposed platform, showcasing how these components interact to create a robust, efficient, and user-centric cloud computing ecosystem.

In the following sections, we will delve deeper into the technical details of each component, exploring how they work together to realize the full potential of decentralized cloud computing.
